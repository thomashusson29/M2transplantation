\section{Introduction}

La transplantation hépatique demeure le traitement de référence des insuffisances hépatiques terminales. Toutefois, la pénurie persistante d’organes disponibles constitue un frein majeur à son développement, avec une mortalité sur liste d’attente estimée entre 10 et 20~\%. Pour répondre à cette situation, le recours à des greffons dits \textit{marginaux} — issus de donneurs âgés, stéatosiques ou ayant subi une ischémie prolongée — s’est considérablement accru au cours des deux dernières décennies. Si cette stratégie augmente le nombre de transplantations possibles, elle s’accompagne d’un risque accru de dysfonction primaire du greffon et de complications post-opératoires sévères.\\

Dans ce contexte, la \textbf{perfusion normothermique ex vivo (NMP)} s’est imposée comme une avancée majeure. En maintenant le foie à 37~°C dans un circuit oxygéné et nutritif, cette technique reproduit un environnement physiologique permettant d’évaluer la viabilité du greffon, de favoriser sa régénération et d’administrer des traitements ciblés avant transplantation. Les essais cliniques récents, notamment ceux de Nasralla \textit{et al.} (2018) et Markmann \textit{et al.} (2022), ont démontré que la NMP réduit les lésions d’ischémie-reperfusion et améliore la qualité des greffons, en particulier ceux issus de donneurs marginaux.\\

Cependant, la perfusion normothermique repose sur l’utilisation de \textbf{culots globulaires humains} provenant des banques du sang. Or, ces produits sont soumis à des contraintes de disponibilité, de coût et de réglementation, constituant un frein à la généralisation de la NMP. Plusieurs alternatives ont été explorées, telles que l’hémoglobine bovine modifiée (\textit{Hemopure\textsuperscript{\textregistered}}) ou l’hémoglobine de ver marin (\textit{HEMO2Life\textsuperscript{\textregistered}}), mais aucune n’a encore prouvé sa sécurité et sa faisabilité clinique en perfusion hépatique.\\

Une approche alternative consiste à \textbf{récupérer le sang issu de la décharge cave abdominale du donneur d’organes} au moment du prélèvement multi-organes. Ce sang, composé du sang circulant du donneur et du liquide de rinçage abdominal, peut être traité par un dispositif de récupération peropératoire (\textit{CellSaver\textsuperscript{\textregistered}}) afin d’obtenir un \textbf{concentré globulaire du donneur (CGD)}. Ce produit pourrait constituer un transporteur d’oxygène autologue, immédiatement disponible et exempt de contraintes transfusionnelles, offrant ainsi une alternative prometteuse aux culots globulaires de banque.\\

Le présent projet vise à \textbf{évaluer la faisabilité et la sécurité d’utilisation du CGD} en perfusion normothermique \textit{ex vivo}, par comparaison aux culots standards. Les analyses porteront sur le profil inflammatoire, le degré d’hémolyse et la qualité microbiologique des concentrés obtenus, afin d’attester leur conformité aux critères européens de sécurité transfusionnelle. À terme, cette approche pourrait réduire la dépendance aux produits sanguins de banque, favoriser la diffusion de la perfusion normothermique en transplantation hépatique, et ouvrir la voie à son extension vers d’autres organes tels que le rein ou le cœur.\\